\chapter{Introduction}

While the web is without doubt the largest collection of human knowledge, it is largely constituted by unstructured information  -- web pages with text, images and other media embedded within them. This organization (or the lack of) of information, poses a difficulty in {\color{red}{TODO}}. 


To remedy some of these perceived challenges, there is an ongoing effort to build and maintain structured knowledge bases. As of today, there exist several of such knowledge repositories, for example: Freebase, DBpedia, Wikidata, etc. Apart from those freely accessible, several internet businesses like Google. Yahoo!, and Baidu maintain their own proprietary knowledge bases, which have been used successfully in assisting internet web search applications.

In the wake of increasing popularity structured data and  in order to provide a common platform for data exchange the World Wide Web Consortium decided to collect these efforts under the umbrella of Semantic Web. The Semantic Web is a an extension of the Web through standards  that promote common data formats and exchange protocols on the Web, most fundamentally the Resource Description Framework (RDF).



\section{Outline}

We continue our exposition to the subject in Chapter 2, where we cover essential background. In Chapter 3, we present a survey of existing approaches that have been attempted towards enabling querying on structured knowledge bases. In Chapter 4, we describe in full detail a novel approach towards addressing the shortcomings of techniques presented in Chapter 3.