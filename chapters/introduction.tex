\chapter{Introduction}

While the web is without doubt the largest collection of human knowledge, it is largely constituted of unstructured information  -- web pages with text, images and other media embedded within them. This organization (or the lack of) of information, poses certain difficulties and inherent limitations in building automated systems that can consume and process information. 

To remedy some of the perceived challenges, there is an ongoing effort to build and maintain structured knowledge bases. As of today, there exist several of such knowledge repositories, for example: Freebase, DBpedia, Wikidata, etc. Apart from those freely accessible, several internet businesses like Google. Yahoo!, and Baidu maintain their own proprietary knowledge bases, which have been used successfully in assisting internet web search applications.

In the wake of increasing popularity structured data and  in order to provide a common platform for data exchange the World Wide Web Consortium decided to collect these efforts under the umbrella of Semantic Web. The Semantic Web is a an extension of the Web through standards  that promote common data formats and exchange protocols on the Web, most fundamentally the Resource Description Framework (RDF).

Even though the attempt is towards giving \emph{structure} to data, the traditional concepts of relational schema and relational data model are eschewed. The data is most commonly modelled as a graph of entities and objects, with an accompanying schema and an ontology. This means that well developed ideas of querying on relational databases can no longer be directly applied; and towards this end query languages such as SPARQL and corresponding query processing engines have emerged.

An important feature of modern knowledge bases is their vastness. Knowledge bases with as many as a billion facts are widely available. While SPARQL and other structured query languages have a number of strong points, their usability is hindered by the limited schema knowledge available to an end user. Therefore, approaches akin to keyword querying in databases, that abstract out schema requirements from the user, have become increasingly relevant.

In this work, we shall survey a number of such keyword querying approaches on structured knowledge bases. We shall also highlight a common weakness in all of them---insufficient performance on complex queries that involve several \emph{joins}. Our own approach towards keyword querying will aim to address this limitation of existing approaches.

\section{Outline}

As a prerequisite to begin our exposition, we provide  an overview of RDF/SPARQL specifications and related concepts in Chapter 2. In Chapter 3, we survey a few select works whose motivation is similar to ours. We highlight important ideas that can be incorporated in building future systems. Finally, Chapter 4 shall provide a detailed discussion of our proposed approach towards the problem. We conclude with the summary and directions for future work in Chapter 5.
