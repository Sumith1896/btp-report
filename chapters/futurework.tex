\chapter{Conclusion and Future Work}

The Entity Relationship Keyword Querying (ERKQ\footnote{Henceforth shall be referred to as ERKQ in all citations.}) framework, provides a powerful combination of ease-of-use and capability of processing complicated queries beyond the scope of existing systems. The former feature stems from the fundamental decision of permitting the user to query using free-keywords, albeit expecting it in a pre-segmented form. This pre-segmentation, along with a carefully crafted query interpretation layer, gives the framework the latter feature.

Staying within the ERKQ model, there are multiple additions possible in the query processing mechanism that can improve the overall performance of the system. 

Recall that the query processing mechanism was described as to be involving the repetition of two actions (predicate selection and entity selection) for various elements (predicates and entities) of the query. It is not clear however, the order of actions that needs to be taken. We believe that great improvements are possible with work towards this end.

The idea of repeating the two selections (say alternatively) is a simple one. Two (or more) selection steps could be combined into one. Even the idea of computing the joins at the end could be extended to computing intermediate joins and prompting the user for more intermediate selections. In approaches like the latter, there is always the inherent trade-off between ease-of-use (or user-discontent) and accuracy/precision of results.

Within the query interpretation layer, we defined the various \emph{potentials} in the most obvious way possible. Sophisticated potential functions need to be designed and their effectiveness needs to be tested.

Lastly, we refrained from mentioning that the narrated system has not been implemented. A complete implementation of the ideas contained in this work is the most critical requirement of the hour, for there are far more ways in which an implementation could fail than it could on paper.